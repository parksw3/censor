
<<<<<<< HEAD
In December 2019, a cluster of pneumonia cases of unknown etiology was reported in China.
The disease, now referred to as the coronavirus disease 2019 (COVID-19), has since affected more than ??? countries.
As of XXX, more than XXX deaths have been confirmed.

\swp{Skipping introduction for now.}

\jd{This part of intro can be short; we no longer really need to remind people what COVID19 is.}

Understanding the time delays between key epidemiological events, is a key component of statistical and modeling efforts to predict and control disease outbreaks. 
These events can be compared within an infected individual (e.g., the incubaton period is the time between infection and symptom onset) or between infected individuals (e.g., the serial interval is the time between symptom onsets of an infector and an infectee).
However, the estimation of these epidemiological delay distributions depends on having observed both events;
this dependency can bias the estimates during an early growth phase of an outbreak.

\section{Theoretical framework}

We begin by modeling epidemiological delays from a cohort perspective.
A ``cohort'' consists of all individuals whose first epidemiological event of interest occurred at a given time.
For example, for the purpose of measuring the symptomatic period, cohort $s$ consists of all individuals who became symptomatic at time $s$.

\jd{Is it going to be better to define a cohort for every epidemiological event, rather than just the ones that \emph{start} an interval? I'm thinking of how useful backwards GIs have been.}
\jd{On the other hand, I like some of the simplicities below.}

Observed delay distributions are generally subject to right-censoring.
Since events must occur before the time of measurement to be observed, delays in cohort $s$ that are longer than $t-s$ cannot be observed at time $t$.
Therefore, the cohort delay distribution $c_s(\tau|t)$ can be expressed as a truncated distribution:
\begin{equation}
c_s(\tau|t) = \frac{f_s(\tau)}{F_s(t-s)},\quad \tau \leq t-s
\end{equation}
where $f_s(\tau)$ is the true delay distribution (which can vary across cohorts) and $F_s(\tau)$ is the corresponding cumulative distribution function.

Other biases also need to be accounted for. Delay distributions involving more than one individual typically change with disease dynamics. For example, serial intervals will be shorter when the number of susceptibles is decreasing rapidly, since there will be fewer infection opportunities as the cohort grows older.
Within-individual delay distributions that depend on external factors (e.g., time between symptom onset and hospitalization) can also change over time.
=======
